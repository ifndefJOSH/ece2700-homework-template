% Engineering style guide LaTeX template
% Created by Josh Jeppson
% You are free to distribute and use this template as long as I am given credit
% Maybe it will make my job of grading your papers easier

\documentclass[letterpaper]{report}
\usepackage{fancyhdr}
\usepackage{tcolorbox}
\usepackage{layout}
\usepackage[margin=1in]{geometry}
\usepackage{pgf}
\usepackage{pgfpages}
\usepackage{pgfplots}
\usepackage{graphicx}
\usepackage{amsfonts}
\usepackage{amsmath}
\usepackage{amssymb}
\usepackage{tikz,pgfplots}
\usepackage[abspage,user,lastpage]{zref}
\usepackage{lastpage}
\usepackage{marginnote}
\usepackage{eso-pic}
\usepackage{ulem}
\usepackage{tikz}
\usepackage{circuitikz}
\usepackage{wrapfig}
\usepackage[export]{adjustbox}
\usepackage{listings}
\usepackage{xcolor}
\usepackage{soul}
\usepackage{cite}
\usepackage{url}
\usepackage[absolute]{textpos}
\usepackage{amssymb}  % provides Rbb, etc.
\usepackage{latexsym}  % provides \Box
\usepackage{array}
% \usepackage{forest}

\pgfplotsset{compat=1.18}

\usetikzlibrary{plotmarks}
\usetikzlibrary{arrows}
\usetikzlibrary{patterns}
\usetikzlibrary{shapes.geometric}
\usetikzlibrary{shapes.arrows}
\usetikzlibrary{decorations}
\usetikzlibrary{decorations.pathreplacing}
\usetikzlibrary{positioning}

\definecolor{codegreen}{rgb}{0,0.6,0}
\definecolor{codegray}{rgb}{0.5,0.5,0.5}
\definecolor{codepurple}{rgb}{0.58,0,0.82}
\definecolor{backcolour}{rgb}{0.95,0.95,0.92}

\lstdefinestyle{mystyle}{
	backgroundcolor=\color{white},
	commentstyle=\color{green},
	keywordstyle=\color{blue},
	numberstyle=\tiny\color{codegray},
	stringstyle=\color{red},
	basicstyle=\ttfamily\footnotesize,
	breakatwhitespace=false,
	breaklines=true,
	captionpos=b,
	keepspaces=true,
	numbers=left,
	numbersep=5pt,
	showspaces=false,
	showstringspaces=false,
	showtabs=false,
	tabsize=2
}
\lstset{style=mystyle}

\allowdisplaybreaks

\usetikzlibrary{arrows.meta}

% ROBDD Style
% \forestset{
%   BDT/.style={
%     for tree={
%       if n children=0{}{circle},% use a circle unless there are 0 children
%       draw,% draw every node
%       edge={
%         my edge,% use the my edge style for edges (with the arrow)
%       },
%       if n=1{
%         edge+={0 my edge},% if the child is the first one, add the 0 my edge style (dashed)
%       }{},
%       font=\sffamily,% use sans serif for node text
%     }
%   },
% }

\usetikzlibrary{calc, positioning, tikzmark}

\newcommand\BackgroundPic{%
\put(0,0){%
\parbox[b][\paperheight]{\paperwidth}{%
\vfill
\centering
\includegraphics[width=\paperwidth,height=\paperheight,%
keepaspectratio]{background.jpg}%
\vfill
}}}

\usetikzlibrary{shapes.geometric, arrows}
\usetikzlibrary{shapes}
\newenvironment{question}{}{}

% Definitions for flowchart stuff
\tikzstyle{rect} = [rectangle, text centered, draw=black, fill=white]
\tikzstyle{circ} = [circle, text centered, draw=black, fill=white]
% \tikzstyle{process} = [rectangle, minimum width=3cm, minimum height=1cm, text centered, draw=black, fill=orange!30]
% \tikzstyle{decision} = [diamond, minimum width=3cm, minimum height=1cm, text centered, draw=black, fill=green!30]
% \tikzstyle{loop} = [signal, minimum width=2cm, minimum height=1cm, text centered, draw=black, fill= blue!30]
% \tikzstyle{predefinedprocess} = [rectangle split, rectangle split horizontal,rectangle split parts=3,minimum height=1cm, draw=black, fill=orange!30]
\tikzstyle{arrow} = [thick,->,>=stealth]

%% ---- Document Properties ----
\newcommand{\course}{ECE XXXX} % Edit these properties
\newcommand{\studentName}{LASTNAME, FIRSTNAME}
\newcommand{\aNumber}{A00000000}
\newcommand{\assnNumber}{1}
\newcommand{\laplace}[1]{\mathcal{L}\left[#1\right]}
\newcommand{\ilaplace}[1]{\mathcal{L}^{-1}\left[#1\right]}
\newcommand{\fourier}[1]{\mathcal{F}\left[#1\right]}
\newcommand{\ifourier}[1]{\mathcal{F}^{-1}\left[#1\right]}

% \newcommand{\diff}[1]{\frac{d}{d#1} \left[ }{\right]}

\newcommand{\vlineStrut}{\rule[-.3\baselineskip]{0pt}{\baselineskip}}
\newcommand{\finalAns}[2]{\tikzmark{as0} \uuline{#1 }  \tikzmark{as}
\marginnote{\tikzmark{rm} \qquad #2}
\qquad $<$ \quad \noindent\makebox[\linewidth]{\rule{6.8in}{0.4pt}}
% \begin{tikzpicture}[overlay, remember picture]
%     \coordinate[below=0.5ex of $(pic cs:as)!0.5!(pic cs:as)$]  (a);
%     \draw (a) -- ++ ( 14,0) node[right] {};
% \end{tikzpicture}
}

% ---- Header and footer def ----
\geometry{
	right=1 in
}
\pagestyle{fancy}
\fancyhead[]{}
\fancyfoot[]{}
\fancyheadoffset[R]{.7in}
\rhead{ \studentName \qquad  \thepage\ / \pageref{LastPage} }
\chead{\vlineStrut \qquad \course; Assignment \assnNumber \qquad \vlineStrut}
\lhead[]{\today }
\renewcommand{\headrulewidth}{0pt}

% ---- Custom commands ---
% \newcommand{\pageline}[][]{\noindent\makebox[\linewidth]{\rule{6.8in}{0.4pt}}}

% ---- Custom Environments ----
\newcounter{problemNum}[section]
\newenvironment{problem}[1][]
{

	% \noindent\makebox[\linewidth]{\rule{6.5in}{0.4pt}} \\
	\noindent \textbf{Problem #1 \refstepcounter{problemNum}:} %(#1)
}
{
	\newline \noindent\makebox[\linewidth]{\rule{6.8in}{0.4pt}}
}

\newenvironment{answer}{
	\noindent
}
{

	\noindent\makebox[\linewidth]{\rule{6.8in}{0.4pt}} \\
	% [.2pt] \\
	\noindent\makebox[\linewidth]{\rule{6.8in}{0.4pt}}
}

\newenvironment{answersection}{}{
	\newline \noindent\makebox[\linewidth]{\rule{6.8in}{0.4pt}}
}





\begin{document}
\AddToShipoutPicture{\BackgroundPic}

\begin{problem}[{1}]
	% Math problem demo
	Solve the following equation for $x$:
	\[
		4 = x + 2
	\]
    Given: Equation \\
    Find: Solution for $x$ % Math statements, even if a single character, are to be included in a $$, \[\], or an align
\end{problem}
\begin{answer}
    % Because this is a math problem, we will use the align* environment.
    % The ampersand, &, indicates what to align to, generally the equals signs
    % align numbers equations, but align* does not
	\begin{align*}
		4 &= x + 2 \\ % We must explicitly state we are starting a new line with \\
		4 - 2 &= x \\
		2 &= x % Do not include a \\ on the last line
	\end{align*}
% Now that we have found the answer, we must double underline it. This can be achieved using the custom \finalAns{}{} command
\finalAns{$2$}{$x$}
\end{answer}
\newpage % This is optional, but since all of this is digital, we can use as many pages as we want!
\begin{problem}[{2}]
	% Problem Demonstrating multiple parts as well as the lstlisting environment
	Write a ``Hello World'' program in the following languages:
	\begin{enumerate}
		\item C
		\item C++
		\item Python
	\end{enumerate}
    Given: Languages \\
    Find: ``Hello World'' program in each one
\end{problem}
\begin{answer}
    % We will use an answersection environment to accomplish this.
    % Since the ``find'' is a code environment, we DO NOT need to (and should not) include it in a \finalAns{}{} command
    \begin{answersection}
		\textbf{\texttt{Hello World} in the C language:}
		\begin{lstlisting}[language=c]
/*
 Hello World in C. This program demonstrates syntax highlighting in the LaTeX lstlisting environment.
 Note how this comment is green
*/
#include <stdio.h> // % Note how even LaTeX comments show up in here

int main(void)
{
	printf("Hello World!\n");
	return 0;
}
		\end{lstlisting}
		\quad % If you get the error: There's no line here to end. \end{answersection}, place a \quad or \qquad (tab character), to make the document compile
    \end{answersection}
	\begin{answersection}
	 \textbf{\texttt{Hello World} in the C++ language:}
	 \begin{lstlisting}[language=c++]
#include <iostream>

int main(void)
{
	std::cout << "Hello World!" << std::endl; // Stream operators are weird, aren't they?
	return 0;
}
	 \end{lstlisting}
	\quad % See, it's kind of annoying, isn't it?
	\end{answersection}
	% The last part of a problem SHOULD NOT be in an answersection, or it will create an extra line
	\textbf{\texttt{Hello World} in the Python language:}
	\begin{lstlisting}[language=python]
if __name__=='__main__':
	print("Hello World!") # LaTeX accurately adapts the syntax highlighting to the language in use!
	\end{lstlisting}

\end{answer}
\newpage % Again, optional
\begin{problem}[{3}]
	Explain why you should get an ``A''. \\ % Note, that in LaTeX, `` and '' are the commands for (proper) quotation marks
    Given: Problem statement \\ % Sometimes, this is a totally valid ``given''
    Find: Explaination
\end{problem}
\begin{answer}
    I should get an ``A'' because I know \emph{so} much \LaTeX! I can do \texttt{monospace} and \textbf{bold} and \textit{italic} text! I can write the quadratic formula in math mode ($x = \frac{-b \pm \sqrt{b^2 - 4ac}}{2a}$), and I can put the current date (\today).
    I can
    \begin{itemize}
		\item create
		\begin{itemize}
			\item lists
			\item within
		\end{itemize}
		\item lists.
    \end{itemize}
	I can also
	\begin{enumerate}
		\item create numbered lists
		\item and use \texttt{makeHW.py}
	\end{enumerate}
	I can indicate a \finalAns{final answer,}{ANS}
	and I know how to create a
	\[
		c_{entered} = equation\left(\frac{with\cdot as\cdot many\cdot symbols}{as\cdot we \cdot need}\right)
	\]
	I can do
	\begin{center}
		centered text,
	\end{center}
	\begin{flushright}
		right-aligned text,
	\end{flushright}
	\begin{flushleft}
		and left-aligned text.
	\end{flushleft}
	I can create a matrix like this:
	\[
		\begin{bmatrix}
			a & b & c \\
			d & e & f \\
			g & h & i
		\end{bmatrix}
	\]
	and a table like this:
	\begin{center}
		\begin{tabular}{|c|c|c|} \hline
			\textbf{Header 1} & \textbf{Header 2} & \textbf{Header 3} \\ \hline
			a & b & c \\
			d & e & f \\
			g & h & i \\ \hline
		\end{tabular}
	\end{center}
	Maybe I'll create my next resume in \LaTeX!
\end{answer}
        
\end{document}
